\documentclass[12pt]{article}
\usepackage[utf8]{inputenc} % Required for inputting international characters
\usepackage[T1]{fontenc} % Output font encoding for international characters
\usepackage{mathpazo} % Palatino font
\usepackage{graphicx}
\usepackage{imakeidx}
\usepackage{url}
\makeindex[columns=3, title=Alphabetical Index, intoc]

\begin{document}

	\begin{titlepage} % Suppresses displaying the page number on the title page
	\begin{center} % Centre everything on the page

		%------------------------------------------------
		%	Headings
		%------------------------------------------------
		
		\textsc{\LARGE Università degli studi di Padova}\\[1.5cm]
		
		\textsc{\Large Corso di Laurea in Informatica}\\[0.5cm]
		
		\textsc{\Large Progetto di Tecnologie Web}\\[0.5cm]
		
		\includegraphics[scale=1.5]{img/logo_original.png}
		
		\textsc{\large a.a. 2018/2019}\\[0.5cm] % Minor heading such as course title
		
		%------------------------------------------------
		%	Title
		%------------------------------------------------
		
		{\huge\ Relazione}\\[0.4cm] % Title of your document
		
		%------------------------------------------------
		%	Author(s)
		%------------------------------------------------
		
	 	\large
			\begin{tabular}{l r}
				\emph{Team:} & \emph{Matricola:} \\
				Giacomo \textsc{Barzon} & 1143164 \\
				Francesco \textsc{De Filippis} & 1143408 \\
				Giacomo \textsc{Greggio} & 1142951 \\
				Michele \textsc{Roverato} & 1143030 \\
			\end{tabular}
	\end{center}
\end{titlepage}
	%----------------------------------------------------------------------------------------
	\section*{Informazioni logistiche}
	
	Il sito è accessibile al seguente link:
	\begin{center}
		\url{http://testsitotecweb.altervista.org/}
	\end{center}

	Account non amministratore utilizzabile:
	\begin{itemize}
		\item \textbf{Username}: linus@torvalds.com
		\item \textbf{Password}: git
	\end{itemize}

	Account amministratore utilizzabile:
	\begin{itemize}
		\item \textbf{Username}: admin@admin.it
		\item \textbf{Password}: admin
	\end{itemize}

	\section{Analisi dei requisiti}
		\subsection{Requisiti}
		Lo scopo di questo progetto consiste nel dare vita ad una piattaforma web che offre la possibilità di consultare un archivio di informazioni sugli argomenti del mondo informatico. Gli utenti troveranno le informazioni suddivise nel seguente modo:
		\begin{itemize}
			\item Nella home page sono disponibili delle card che descrivono il macroargomento che trattano con relativa immagini e alcuni link ai principali sottoargomenti;
			\item I sottoargomenti, a cui è dedicata una pagina, contengono dei link che si riferiscono agli articoli facenti parte di quel particolare sottoargomento.
		\end{itemize}
		 L'utente ha inoltre la possibilità di commentare e votare i singoli articoli, interagendo così con gli altri utenti registrati al blog. Gli amministratori dispongono di un pannello amministratore in cui vengono elencate le operazioni che possono compiere, tra cui aggiornare la piattaforma con nuovi contenuti ed eventualmente migliorare contenuti già esistenti.
		IDevono inoltre occuparsi di moderare l'ambiente sfruttando la possibilità di eliminare eventuali commenti inopportuni.
	\subsection{Target utenza}
		Il sito è rivolto ad un pubblico che ha dimestichezza con la tecnologia e certamente con la ricerca e l'utilizzo di Internet per trovare informazioni. Inoltre può essere consultato da persone che desiderano acquisire delle nozioni sui principali argomenti riguardanti l'informatica, perciò si prevede che la piattaforma verrà raggiunta da utenti con età superiore ai 14 anni (e.g. dagli studi superiori in poi).
		
	\section{Architettura delle informazioni}
	\subsection{Descrizione}
		In questa sezione verrà descritto il modo in cui è stato scelto di gestire le informazioni presentate nel sito. Poichè si tratta di un blog in cui la quantità di contenuti è destinata a crescere nel tempo è stata adottata un'interfaccia a tre pannelli, navbar, sidebar e contenuto. La spiegazione dettagliata verrà fornita nella \textbf{Sezione \ref{Usabilità} (Usabilità)}.
	\subsection{Struttura pagine pubblica}
		La parte pubblica del sito è accessibile a tutti gli utenti e di seguito verranno descritte le pagine visitabili con le relative funzionalità che queste offrono.
		\begin{itemize}
				\item \textbf{Homepage (index.php)}: si tratta della pagina principale del sito in cui l'utente trova le seguenti informazioni:
				\begin{itemize}
						\item \textbf{Navbar}: parte superiore della pagina in cui viene mostrato il logo del sito e una lista orizzontale di link utili alla navigazione;
						\item \textbf{Header}: immagine si sfondo della parte sottostante alla navbar e nome del sito;
						\item \textbf{Descrizione e Contenuto}: nella parte inferiore all'header è presente una descrizione del sito, seguita dalla lista dei macroargomenti visualizzati sottoforma di Card;
						\item \textbf{Footer}: parte inferiore del sito contenente ulteriori le informazioni sulla piattaforma.
				\end{itemize}
		\end{itemize}
	\subsubsection{Struttura pagine privata}
	Bla bla bla

	\section{Usabilità}
	\label{Usabilità}
	\subsection{Descrizione}
	asds
	\section{Layout e design}

	\section{Accessibilità}
	\subsection{Introduzione}
	dsa
	\subsection{Scelte adottate}
	das
	\subsubsection{Colori}
	dsa
	\subsubsection{Link}
	dsa
	\subsubsection{JavaScript}
	da
	\subsubsection{Breadcrumb}
	dsa
	

	\section{Sviluppo}
	\subsection{Tecnologie utilizzate}
	sada
	\subsection{Strumenti di supporto}
	travis
	\subsection{Form}
	\subsubsection{Struttura dei form}
	sa
	\subsubsection{Validazione dei form}
	sa, sezioni per js, php, html ecc...
	\subsection{Separazione parte privata e pubblica}
	asdsa
	\subsection{Struttura pannello amministratore}
	asds
	\subsection{Classi}
	sadsd
	\subsection{Gestione degli errori}
	asdsdas
	\subsection{Progettazione del Database}
	sasa
	
	\section{Scelte implementative}
	\subsection{Introduzione}
	sajdas
	\subsection{Scelta riguardo ...}
	sadd
	\subsection{Scelta riguardo ...}
	asdsad
	
	\section{Compatibilità}
	
	\section{Test}
	\subsection{Accessibilità}
	dsa
	\subsubsection{Strumenti utilizzati}
	asd
	\subsubsection{Risultati}
	das
	
	\subsection{Usabilità}
	dsa
	\subsubsection{Strumenti utilizzati}
	dsa
	\subsubsection{Risultati}
	asd
	
	\subsection{Performance}
	dsa
	\subsubsection{Strumenti utilizzati}
	sda
	\subsubsection{Risultati}
	asd

	\subsection{Validazione}
	dsa
	\subsubsection{Strumenti utilizzati}
	das
	\subsubsection{Risultati}
	dsad
	
	\section{Suddivisione dei ruoli}
	
	\begin{itemize}
		\item \textbf{Francesco De Filippis}: 
		\item \textbf{Giacomo Barzon}:
		\item \textbf{Giacomo Greggio}:
		\item \textbf{Michele Roverato}:
	\end{itemize}

\end{document}
